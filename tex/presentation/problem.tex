\section{Problématique et modèle}

\begin{frame}{Contraintes sur la géométrie}
  \vfill 
  \begin{itemize}
  \item $N_t$ : largeur de la travée $t$.
  \end{itemize} 
  \vfill
  Contraintes : 
  \vfill
  \begin{itemize}
  \item $1 \leq N_t \leq 15$,
    \vfill
  \item $\mathrm{Card}(N_t\vert N_t \leq  6) \geq  P$,
    \vfill
  \item la largeur d'une allée est supérieure à $15m$.
    \vfill
  \end{itemize}   
  \vfill
\end{frame}

\begin{frame}{\'Etat de l'art}
  \begin{itemize}
  \item dynamique \cite{rekik2015}
  \item génétique/aéroport \cite{gotteland2004}
  \item optimisation de forme/level set \cite{allaire2006}
  \item combinatoire \cite{moussi2012}
  \item heuristique \cite{ndiaye2015}
  \item Kim Kim \cite{kim98}
  \item geometrical and functional layout optimization/Jacquenot \cite{jacquenot2010}
  \end{itemize}
\end{frame}

\begin{frame}{Algorithme}
  \resizebox{\textwidth}{!}{
  \begin{tikzpicture}%[scale=0.2]
    \draw (0,0) node[draw,rectangle,text width=4cm,align=center,fill=blue!20] {Génération de la configuration};
    \draw (6,0) node[draw,rectangle,text width=4cm,align=center,fill=blue!20] {Calcul du coût};
    \draw (11,0) node[draw,diamond,fill=red!10]{Optimal?};
    \draw (14,0) node[draw,rectangle,fill=green!30]{Fin};
    \draw[->,thick] (2.1,0) -- (3.9,0);
    \draw[->,thick] (8.1,0) -- (9.8,0);
    \draw[->,thick] (12.2,0) -- (13.6,0);
    \draw (12.9,0) node[above] {Oui};
    \draw[->,thick] (2.1,0) -- (3.9,0);
    \draw[->,thick] (11,1.2) -- (11,3) -- (0,3) -- (0,0.55);
    \draw (5.5,3) node[above] {Non};
  \end{tikzpicture}}
  \vfill
  \emph{Objectif : } minimiser la fonction coût en fonction de la \emph{géométrie} et de la \emph{configuration}.
  \vfill
  L'idéal serait de pouvoir générer des géométries et des configurations, puis de comparer les différentes valeurs des fonctions coût. 
  \vfill  
\end{frame}

\begin{frame}{Hypothèses}
  \begin{itemize}
  \item On se concentre seulement sur le cas des conteneurs vides
    \vfill
  \item On ne considère que le nouveau portique
    \vfill
  \item On se concentre seulement sur la zone entre les camions et le nouveau portique
    \vfill
  \item On ne considère que les itinéraires bateau-stockage-camion
    \vfill
  \item Les conteneurs sont stockés par blocs de hauteur 5
    \vfill
  \item Ils sont déchargés dans une travée par une allée et sont chargés par l'allée à l'autre bout de la travée
    \vfill
  \item Les travées ne contiennent des conteneurs appartenant à un seul client
    \vfill
  \item On ne considère que des conteneurs de taille identique (20 pieds).
  \end{itemize}
  \vfill
\end{frame}

\begin{frame}{Fonction de coût}
  \vfill
  La fonction coût sert à comparer les différentes configurations, c'est-à-dire la géométrie et la répartition des clients dans les travées.\\
  \vfill
  La fonction coût cherche à évaluer les coûts moyens liés aux emplacements des 4500 EVP présent en moyenne dans le port.
  Cette fonction coût tient compte des déplacements entrée-sortie en fonction de la fréquence de déplacement des conteneur qui peut dépendre de chaque client.
  \vfill
  On attribue donc un poids à chaque catégorie (client+dimension), représentant la probabilité que les travées attribuées à cette catégorie soient visitées.
\end{frame}

\begin{frame}{Fonction de coût}
  La fonction coût correspond à la distance moyenne parcourue par un conteneur.
  \begin{description}
  \item[$d_t$] distance associée à une travée $t$ pour une géométrie fixée.
  \item[$\mu_c$] poids du client $c$,
  \item[$x_{t,c}$] nombre d'emplacements utilisés par le client dans la travée $t$.
  \end{description}
  \vfill
  La fonction coût  : 
  $$ \sum_t \sum_c d_t\mu_c x_{t,c}.  $$
  \vfill
\end{frame}

\begin{frame}{Poids des clients}
  Extraits à partir de données fournies sur 2 mois : 
  \vfill
  \begin{description}
  \item[$V_{tot}$] nombre total de mouvements de conteneurs sur le port,
  \item[$V_c$] nombre de mouvements de conteneurs par le client $c$.
  \end{description}
  \vfill
  Le poids du client est donné par:
  \begin{equation*}
    \mu_c = \frac{V_c}{V_{tot}}
  \end{equation*}
\end{frame} 

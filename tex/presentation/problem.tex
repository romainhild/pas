\section{Problématique et modèle}

\begin{frame}{\'Etat de l'art}
  \begin{itemize}
  \item dynamique \cite{rekik2015}
  \item aéroport \cite{gotteland2004}
  \item optimisation de forme/level set \cite{allaire2006}
  \item combinatoire \cite{ndiaye2015}
  \item heuristique
  \item génétique
  \item Kim Kim \cite{kim98}
  \item Jacquenot \cite{jacquenot2010}
  \end{itemize}
  \vfill
  L'objectif est de minimiser la fonction coût en fonction de la géométrie et de la configuration.
  \vfill
  L'idéal serait de pouvoir générer des géométries et des configurations, puis de comparer les différentes valeurs des fonctions coût. 
  \vfill  
\end{frame}

\begin{frame}{Hypothèses}
  \begin{itemize}
  \item On se concentre seulement sur le cas des conteneurs vides
    \vfill
  \item On ne considére que le nouveau portique
    \vfill
  \item On se concentre seulement sur la zone entre les camions et le nouveau portique
    \vfill
  \item On ne considère que les itinéraires bateau-stockage-camion
    \vfill
  \item Les conteneurs sont stockés par blocs de hauteur 5
    \vfill
  \item Ils sont déchargés dans une travée par une allée et sont chargés par l'allée à l'autre bout de la travée
    \vfill
  \end{itemize}
  \vfill
\end{frame}

\begin{frame}{Fonction de coût}
  \vfill
  \begin{itemize}
  \item  La fonction coût  : 
    \vfill
    $$ \sum_t \sum_c d_t* \mu_c* x_{t,c}.  $$
    \vfill
  \item La fonction coût correspond à la distance moyenne parcourue par un conteneur.
  \end{itemize}
  \vfill
\end{frame}

\begin{frame}{Poids des clients}
  Extraits à partir de données statistiques : 
  \begin{itemize}
    \vfill
  \item $\mu_c$ : poids du client $c$,
    \vfill
  \item $V_{tot}$ : nombre total de conteneurs (4500 EVP),
    \vfill
  \item  $V_c$ : nombre "moyen" de conteneurs stoqués sur le port par le client $c$.
  \end{itemize}
  \vfill
\end{frame} 

% \begin{frame}{Quelques termes techniques}
%   \vfill
%   \begin{itemize}
%   \item Travée, pile, bloc, allée, zone de chargement.
%     \vfill
%   \item Géométrie : répartition des emplacements sur le terminal.
%     \vfill
%   \item Client.
%     \vfill
%   \item Configuration : une répartition des conteneurs pour une géométrie fixée.
%   \end{itemize}
%   \vfill
% \end{frame}

% \begin{frame}{Contraintes sur la géométrie}
%   \vfill 
%   \begin{itemize}
%   \item $N_t$ : largeur de la travée $t$.
%   \end{itemize} 
%   \vfill
%   Contraintes : 
%   \vfill
%   \begin{itemize}
%   \item $1 \leq N_t \leq 15$,
%     \vfill
%   \item $\mathrm{Card}(N_t\vert N_t \geq  6) \leq  30$,
%     \vfill
%   \item la largeur d'une allée est supérieure à $15m$.
%     \vfill
%   \end{itemize}   
%   \vfill
% \end{frame}

% \begin{frame}{Géométrie et distances}
%   \vfill
%   \begin{itemize}
%   \item $d_t$ : distance associée à une travée $t$ pour une géométrie fixée.
%     \vfill
%   \item $d_{t,E1}$ : distance entre la travée $t$ et l'entrée $1$.
%     \vfill
%   \item $d_{t,E2}$ :  distance entre la travée $t$ et l'entrée $2$.
%     \vfill
%   \item $d_{t,z}$ :  distance entre la travée $t$ et la zone de chargement.
%   \end{itemize}
%   \vfill
%   $$d_t=\frac{2}{3}d_{t,E1}+\frac{1}{3}d_{t,E2}+d_{t,z}.$$
%   \vfill
% \end{frame}

% \begin{frame}{Configuration}
%   \begin{itemize}
%   \item  $x_{t,c}$ : nombre d'emplacements utilisés par le client dans la travée $t$.
%   \end{itemize} 
%   \vfill Contrainte : 
%   \vfill
%   \begin{itemize}
%   \item $x_{t,c}x_{t,c'}=0,\quad \forall t \quad \forall c\neq c'$ : un travée est utilisée par un seul client.
%   \end{itemize}
%   \vfill
% \end{frame}
